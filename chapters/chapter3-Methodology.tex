\chapter{Methodology}
\label{chap:meth}

This chapter starts with the analysis of requirements for the implemented
software. In the second section of the chapter I will discuss possible
approaches to adapt an external state machine to Phantom OS persistence model.

\section {Requirements analysis}
\label{sec:meth:req}

As previous chapters of this paper suggest, the goal of this paper is to create
an implementation of a TCP suitable for a persistent system. More precisely, I
target a port of Phantom OS to the Genode OS framework. This section contains a
list of requirements that I will follow while designing my own protocol
implementation.

\subsection {Transparency for client applications}
Persistent operating systems (POSs) aim to make development of applications
easier. Persistent operating systems provide their clients with an abstraction
of an uninterrupted execuiton. In a real world, however, a computer system may
experience downtimes. The point of a POS is to hide these downtimes from
applications that use it. Such hiding of undesired effects is generally called
\textit{transparency}.

As discussed in Section \ref{sec:int:phantom}, the current implementation of
the Phantom OS TCP stack is not transparent for the applications residing
in PVM. Application designers should foresee the existence of these errors and
write handlers for them. To make the PoG port really orthogonally persistent
these errors should be handled by the OS, not by client applications.

That means that the state of TCP connections visible for applications should be
the same before a shutdown and after a reboot. To achieve that a persistent TCP
implementation should either (1) backup and restore connections state or (2)
make the disappearance of a remote peer invisible for application on the other
end of the connection. Of course, restoration of the state should be done
before returning control to applications.

(1) can be achieved with reestablishing connections on startup. However, the
remote peer should be ready for the reestablishment request. This is
problematic in cases when a persistent host plays the role of a server in a
client-server communication. (2) can be achieved with use of a message broker
that will mimic an alive host with full TCP input queue. Thus it will keep the
connection alive but prohibit sending data to the channel. Another way to
achieve (2) is to send a control message to the remote socket. The remote TCP
in this case should efficiently do the same thing as a message broker did. It
should mimic a live connection that cannot accept data. The drawback of this
approach is that to exchange control messages both communicating peers should
have enhanced TCP stack. That is, they should support those control messages
and be able to process them.

\subsection {Robustness to rollbacks}

Phantom OS uses full-memory snapshots of PVM memory space to provide
persistence for its clients. Due to this fact, computational operations of PVM
applications can be executed more than once due to restart. For example,
assume that an application receives packets A, B, and C. Assume that after
packet A is received, snapshotting is performed. While it was in progress, the
application receives packet B. After the snapshot was made, applications
receives packet C and immediately after this machine goes down.  After the
system is restored from the snapshot the application will have no knowledge
that it already received packets B and C. Thus the application will send
acknowledgement segment for A, which will result in error because the remote
peer does not expect this segment.  Moreover, it is unable to retransmit this
packet, since it does not exist in remote TCP's retransmission queue, because
it was already acknowledged when the application received B and C.

Phantom OS snapshot mechanism takes into account that snapshot is not an atomic
process. Thus, in the usual case it would handle reception of packet B
properly, i.e. the client application would know that B was received. But in
case of PoG the TCP stack is not stateless like Phantom kernel but still
transient.  Therefore, if TCP receives a packet but does not pass it to the
application before restart, this packet may be lost. The remote peer expects
that the application processed it but the application never received it because
TCP did not finish delivering it to the application.

Three solutions exist that may solve the problem above. The first is to refrain
from sending acknowledgement packets if a received packet is not included in any
snapshot. However, this will either dramatically increase the number of
retransmissions in a network, which will result in reduced network performance
or it will require doing snapshots very often. In such a case a snapshot should
be performed each 50ms to avoid extra network congestion at all. However, in
PoG a snapshotting runs for around 10 seconds. This approach is not acceptable
with such a long snapshotting process.  

The second solution is to keep in persistent memory a buffer that will store
all packets that TCP clients have not consumed yet. In this case persistent
memory can be a regular file or snapshotted memory space. 

The third approach is to have a message broker inside a network that will
function similarly to the buffer in the second approach. The broker can save
all acknowledged packets in a network, and flush its queue when it detects that
the receiving host has done a snapshot. ICMP messages or some similar stateless
network protocol could be useful to notify the broker about the end of the
snapshot process.

\subsection {Usefulness}

Phantom OS network stack is already capable of restoring sockets in CLOSED and
LISTEN states. For a Genode implementation of a persistent TCP to be useful it
should work for the same or bigger set of states. However, even if the
developed software will only support the same set of states the fact that the
software is implemented as a Genode component will greatly improve
maintainability of the software. The software will be useful for the PoG port
which lost all socket enhancements of native Phantom OS.

\subsection {Easy migration and independence of Phantom OS}

The last design goal is not directly related to providing persistent I/O
abstraction for the clients of a network stack. Since snapshotting is a very
popular technique for implementing persistent systems, and Genode is actively
used to develop various operating systems, it is probable that some other
developers might want to use persistent TCP in their software.

That means that the result of this work should be a standalone application that
works without Phantom and is flexible enough to support all use cases that may
arise. Fortunately, Genode's component-based architecture helps greatly with
it. Another reason to make persistent TCP implementation a standalone
application is that the PoG port is not yet ready for me to experiment with it.

Another thing that should be addressed in the design of the API of the system,
is easy migration from non-persistent TCP stacks that are already available in
Genode, namely lwip and lxip. To do this I intend to design the API very
similar to regular sockets or even exactly the same, but with different call
semantics. I also plan to add a Virtual File System (VFS) plugin that will
expose sockets as regular files because this is the way Genode developers
suggest integrating sockets into the Genode ecosystem since version 18.11. With
such implementation, even relinking of existing binary will not be needed. The
only thing a user would need is to change the configuration of the init Genode
component.

\subsection{Application of requirements for the implemented software}

Implementation of a TCP stack that will satisfy all the requirements listed
above is not in the scope of this work. They are listed as requirements for an
ideal network stack for a persistent OS. Instead, I aim to solve a much
simpler problem.

When a local host goes down, the remote TCP still maintains a connection for a
certain amount of time, retransmitting the last unacknowledged segment as
described at Section \ref{lr:tcp}. If a host is shut down for a short amount of
time, TCP will repair itself. This is valid if the state of local TCP will not
be changed due to the restart. But Genode components do not keep their state
after restart. Therefore, to allow TCP reliable delivery service to properly
work with a restarted host system needs to restore the state of TCP stack on a
host that goes down. The practical part of this thesis will be dedicated to
implementation of a mechanism to restore locally-stored state of TCP.

\section{On persisting of finite state machines}
\label{sec:meth:fsms}

Most TCP implementations implement logical state of the sockets as finite state
machines. Since the goal of the paper is to create a persistent TCP
implementation, it might be useful to consider how an arbitrary FSM can be
snapshotted to restore it later to the same state from the snapshot. 

The most straightforward and universal way to achieve the desired outcome is to
use a full-memory snapshot of a running process. Full-memory snapshot should
include stack and address space. Also it should include current context, i.e.
general-purpose registers and the program counter. Obtained snapshot should be
stored in a persistent memory and loaded to the appropriate memory on restore.
In principle, this approach would work for any running process. A drawback of
this approach is that an operating system should provide mechanisms to allow
such manipulations with address spaces.

The second approach is to directly access a current state of the FSM. Usually
most operating systems components, including network stacks, are implemented
on C or C++. Usually FSM in these languages are implemented as structs or
classes holding current state as an value of an enumerable type. If it is
possible to access these structures directly, simple saving and loading them
would work. In some cases, for example sockets, there are several state
machines: one per socket. In such cases FSMs should be registered at the
supervisor that controls inputs and stores FSM structures.

The last approach requires knowing of the internal functioning of an FSM. When
the set of all possible states and transitions of FSM is known in advance, it
is possible to implement FSM with the same transitions. When the tracking FSM
is provided with the same inputs as the target one states of both FSMs should
be in sync. Assessing the state of the tracking FSM will allow an observer to
determine the state of the target FSM.  Moreover, tracking of inputs greatly
helps in the restoration phase. To restore the state of the target FSM it is
enough to provide it with the same inputs once again. This approach is useful
when an FSM does not have a centralized state or when it cannot be easily
accessed.

