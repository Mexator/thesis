\chapter{Introduction}

\section{Background}
\label{sec:intro-bg}

Today, almost any enterprise application operates on data. A part of these data
often should be \textit{persistent}, that is, should outlive the process that
created it. In this definition \textit{a process} is not an operating system 
process, but instead some abstract operation that creates data. 

The division between persistent and non-persistent data is not dichotomic.
Several degrees of persistence exist -- from temporary values during expression
evaluations to long-stored records in a database. For example, when a customer
uses a bank card to purchase goods, the process is the withdrawal of money from
their bank account. This process creates a record about the withdrawal that can
be seen later in a bank app. In the case of expression evaluation, the process
is an execution of a CPU instruction. 

This range of data persistence closely resembles the memory hierarchy of modern
computers. Such similarity is for a reason: more persistent data are stored in
slower and energy-independent upper levels of memory. The presence of the
memory hierarchy brings the need for moving portions of data to lower memory
levels when a program needs to access the data. On the other hand, the limited
size of low memory levels forces programmers to move data that is no longer
needed to the higher levels to make space for new data.

Modern operating systems manage moving data between the main memory and
registers automatically, without the intervention from developers of
applications for those systems. However, when an application needs to access
data from the secondary storage, it should read and load data to the main
memory by itself. Writing a program code for loading and saving data from and to
secondary storage is a burden for developers of that applications.

To remedy this management of moving data between memory layers, Atkinson and
Morrison \cite{atkinson1995orthogonally} propose to use persistent support
systems. These systems are a software for automatic management of physical
memory layers. The systems provide its users with a sandbox to run programs
where all data virtually have the same persistence.

\section{Phantom OS}
\label{sec:int:phantom}

The idea of persistent support systems was developed later with changing focus
to operating systems \cite{landau1992checkpoint,dearle1994grasshopper}. The
most modern of them up to date is Phantom OS. This system consists of a
stateless kernel and a Phantom Virtual Machine (PVM). PVM hosts processes,
which execution states persist across restarts of the host machine. Persistence
is achieved with periodic snapshotting. In the context of Phantom OS, a
snapshot is a memory dump of the PVM memory space. The system uses the latest
snapshot as a recovery point on booting.

Even though Phantom OS is a working persistent support system, there is still a
room for improvement. Currently, Phantom OS does not restore the state of the
Transmission Control Protocol (TCP) stack. The operating system only handles
sockets in the LISTEN state. If a socket was in this state when the snapshot
was taken, the system will reopen it on boot and bind to the same address.
Sockets in other states will become invalid. Any attempt to use these sockets
will result in an error. The application that tries to use such socket will
receive this error and will be responsible for its handling. Usually, that
handling implies reestablishing a connection with a remote peer and restarting
data transmission from the very beginning.

Because of the presence of these errors, existence in a persistent environment
is not completely transparent for Phantom OS applications. Moreover, the common
way of handling these errors, which was described above, results in inefficient
network use, especially when power is off for a short period of time.

\section{Problem statement}

Recently, Phantom OS was ported to the Genode OS framework. The PVM was ported
as a userspace Genode component, and all PVM syscalls are implemented as
functions provided by other Genode parts. In particular, the network stack was
implemented in the Genode as a Virtual File System (VFS) plugin. Due to this
fact and the flexibility of the Genode VFS, changing implementation of the
networking stack became easier in comparison to Phantom OS before this port. My
hypothesis is that it is possible to develop an enhancement to the Phantom
networking stack entirely in the Genode part of the port.

The original document describing TCP \cite{john1981transmission} suggests using
a Finite State Machine (FSM) to implement the protocol. Most implementations of
the protocol, including the one used in Phantom-over-Genode (PoG), use this FSM
technique. The theoretical challenge behind implementation of the enhancement
described above is the integration of an external state machine into the
Phantom persistence model. In this context, “external” means that the code of
the state machine is not running as managed code inside PVM. The first aim of
this paper is to develop a methodology for integrating such state machines into
Phantom OS.  

The second aim of this paper is to enhance PoG TCP stack to achieve two goals.
The first is to reduce the number of errors that should be handled by
applications supervised by the PVM. This means that as many errors as possible
should be either prevented or handled at the Genode layer of the OS. The second
goal is to make the network utilization more efficient. This means avoiding
extra TCP transmissions, if they are not necessary.

The rest of this thesis is structured as follows:
Chapter \ref{chap:lr} contains a detailed description of persistence concept in
and a review of related work. Chapter \ref{chap:meth} describes implementation
details of the PoG port, approaches that were tried during the implementation
and reasons why some of them were discarded in favor of others. Chapter 
\ref{chap:eval} contains description of achieved results, experiments to
evaluate them and a future directions for networking in persistent systems.

