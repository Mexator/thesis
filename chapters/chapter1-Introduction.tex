\chapter{Introduction}

\section{Background}
\label{sec:intro-bg}

Today, almost any enterprise application operates on data. A part of these data
often should be \textit{persistent}, that is, it should outlive the process
that created it. In this definition \textit{a process} is not an operating
system process, but instead a process is an abstract operation that creates
data. 

The division between persistent and non-persistent data is not dichotomic.
Several degrees of persistence exist -- from temporary values during expression
evaluations to long-stored records in a database. For example, when a customer
uses a bank card to purchase goods, the process is the withdrawal of money from
the customer's bank account. This process creates a record about the withdrawal
that can be seen later in a bank app. In case of expression evaluation, the
process is an execution of a CPU instruction. This process creates a data that
is stored in CPU register.

This range of data persistence closely resembles the memory hierarchy of modern
computers. Such similarity is for a reason: more persistent data are stored in
upper levels of memory. These memory levels are often energy-independent and
slower than lower ones. The presence of the memory hierarchy brings the need
for moving portions of data to lower memory levels when a program needs to
access a certain portion of data. On the other hand, the limited size of low
memory levels forces programmers to move data that is no longer needed to the
higher levels to make space for new data.

Modern operating systems and programming lanuages manage moving data between
the main memory and registers automatically, without the intervention from
application developers. However, when an application needs to access data from
a secondary storage, it should read and load data to the main memory by itself.
Writing a boilerplate program code for loading and saving data from and to
secondary storage adds more complexity to programs. Thus, it is a burden for
developers of that applications.

To remedy this management of moving data between memory layers, Atkinson and
Morrison \cite{atkinson1995orthogonally} propose to use persistent support
systems. These systems are a software for automatic management of physical
memory layers. The systems provide its users with a sandbox to run programs
where all data virtually have the same persistence.

\section{Phantom OS}
\label{sec:int:phantom}

The idea of persistent support systems was developed later with changing focus
to operating systems
\cite{landau1992checkpoint,dearle1994grasshopper,atkinson2000review}.
The most modern of them up to date is Phantom OS. This system consists of a
stateless kernel and a Phantom Virtual Machine (PVM). PVM hosts processes,
which execution states persist across restarts of the host machine. Persistence
is achieved with periodic snapshotting. In the context of Phantom OS, a
snapshot is a memory dump of the PVM memory space. The system uses the latest
snapshot as a recovery point on booting.

Even though Phantom OS is a working persistent support system, there is still a
room for improvement. Currently, Phantom OS does not restore the state of the
Transmission Control Protocol (TCP) stack. The operating system only handles
sockets in the LISTEN state. If a socket was in this state when the snapshot
was taken, the system will reopen it on boot, bind to the same address and
start listening on it. Sockets in other states will become invalid. Any
attempt to use these sockets will result in an error. The application that
tries to use such socket will receive this error and will be responsible for
its handling. Usually, that handling implies reestablishing a connection with a
remote peer and restarting data transmission from the very beginning.

Because of the presence of these errors, existence in a persistent environment
is not completely transparent for Phantom OS applications. Moreover, the common
way of handling these errors, which was described above, results in inefficient
network use, especially when power is off for a short period of time. 

\section{Problem statement}

Recently, a group of researchers from Innopolis University started porting
Phantom OS to the Genode OS framework. The PVM in this port is implemented as a
userspace Genode process, and all the Phantom kernel syscalls are implemented
as functions provided by the Genode layer of the port. In particular, the
network stack was implemented in the Genode as a Virtual File System (VFS)
plugin. Due to this fact and the flexibility of the Genode VFS, changing
implementation of the networking stack became easier in comparison to Phantom
OS before this port. 

The functionality of restoration of sockets state was not in the scope of this
port. Thus, even the inherent Phantom OS ability to restore listening sockets
was lost in this port. My hypothesis is that it is possible to develop an
enhancement to the networking stack of Phantom OS port. This enhancement should
reside entirely in the Genode part of the port. The enhancement should provide
client applications with an abstraction of a persistent socket, i.e. socket
that can be used after any number of restarts of a host machine. To simplify
the problem I will implement the mechanism working only for short-duration
shutdowns that gets unnoticed by the remote side. However, I will consider
a cooperative mechanisms that could allow handling longer shutdowns.

The original document describing TCP \cite{john1981transmission} suggests using
one Finite State Machine (FSM) per socket to implement the protocol. Most
implementations of the protocol, including the one used in Phantom-over-Genode
(PoG) port, use this FSM technique. The theoretical challenge behind an
implementation of the enhancement described above is the integration of an
external state machine into the Phantom persistence model. In this context,
“external” means that the code of the state machine is not running as managed
code inside PVM. Thus, the first aim of this paper is to develop a methodology
for integrating such state machines into Phantom OS. 

The second aim of this paper is to enhance PoG TCP stack to achieve two goals.
The first goal is to reduce the number of errors that should be handled by
applications supervised by the PVM. This means that as many errors as possible
should be either prevented or handled at the Genode layer of the OS. The second
goal is to make the network utilization more efficient. This means avoiding
extra TCP transmissions, if they are not necessary.

The rest of this thesis is structured as follows: Chapter \ref{chap:lr}
contains a detailed description of the persistence concept, brief overview of
Phantom OS, the Genode framework and TCP, and a review of related work in
creation of persistent networking stacks. Chapter \ref{chap:meth} establishes
requirements for the implemented software and discusses theoretical approaches
that can be used to implement the software. Chapter \ref{chap:impl} discusses
relevant implementation details of the PoG port, approaches that were tried
during the implementation, and reasons why some of them were discarded in favor
of the others. Chapter \ref{chap:eval} contains a description of the achieved
results, description of experiments to evaluate them and future directions.
Chapter \ref{chap:conc} contains a brief summary of this paper.

